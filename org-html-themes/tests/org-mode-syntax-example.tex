% Created 2024-08-27 Tue 10:14
% Intended LaTeX compiler: pdflatex
\documentclass[11pt]{article}
\usepackage[utf8]{inputenc}
\usepackage[T1]{fontenc}
\usepackage{graphicx}
\usepackage{longtable}
\usepackage{wrapfig}
\usepackage{rotating}
\usepackage[normalem]{ulem}
\usepackage{amsmath}
\usepackage{amssymb}
\usepackage{capt-of}
\usepackage{hyperref}
\author{Fabrice Niessen}
\date{\today}
\title{Org mode syntax example}
\hypersetup{
 pdfauthor={Fabrice Niessen},
 pdftitle={Org mode syntax example},
 pdfkeywords={syntax, org, document},
 pdfsubject={Org mode syntax example},
 pdfcreator={Emacs 29.4 (Org mode 9.6.28)}, 
 pdflang={English}}
\begin{document}

\maketitle
\setcounter{tocdepth}{2}
\tableofcontents

This is an Org mode document.

\textbf{Org mode} is a easy-to-write \emph{plain text} formatting syntax for authoring \LaTeX{}
documents, creating Web pages and much more!

\begin{html}
<script src="\url{http://platform.twitter.com/widgets.js}"></script>
<a href="\url{https://twitter.com/share}" class="twitter-share-button" data-via="f\textsubscript{niessen}">Tweet</a>
\end{html}

\section*{Basics}
\label{sec:org0df0039}

\subsection*{Biggest heading}
\label{sec:org6289322}

New chapter.

\subsubsection*{Bigger heading}
\label{sec:org0a1c795}

New section.

\paragraph*{Big heading}
\label{sec:org3a92b3f}

New sub-section.

\paragraph*{Text breaks}
\label{sec:org7655a77}

A single newline has no effect.
This line is part of the same paragraph.

But an empty line

demarcates paragraphs.

By entering two consecutive backslashes,
you can force to break lines \\[0pt]
without starting a new paragraph.

For an horizontal line, insert at least 5 dashes: this is some text above an
horizontal rule

\noindent\rule{\textwidth}{0.5pt}
and some text below it.

\paragraph*{Numbered headings}
\label{sec:org7a47aca}

You can create numbered headings up to a certain level by setting an option:

\begin{verbatim}
#+OPTIONS: H:4
\end{verbatim}

\subsubsection*{Text width}
\label{sec:orgf9a0e2b}

One morning, when Gregor Samsa woke from troubled dreams, he found himself
transformed in his bed into a horrible vermin. He lay on his armour-like back,
and if he lifted his head a little he could see his brown belly, slightly domed
and divided by arches into stiff sections. The bedding was hardly able to cover
it and seemed ready to slide off any moment. His many legs, pitifully thin
compared with the size of the rest of him, waved about helplessly as he looked.

\subsection*{Lists}
\label{sec:orgce66284}

Org markup allows you to create bulleted or numbered lists. It allows any
combination of the two list types.

\subsubsection*{Unordered lists}
\label{sec:org9f48376}

Itemized lists are marked with bullets. They are convenient to:

\begin{itemize}
\item organize data, and
\item make the document
\begin{itemize}
\item prettier, and
\item easier to read.
\end{itemize}
\end{itemize}

Create them with a minus or a plus sign.

\subsubsection*{Ordered lists}
\label{sec:org24bbd14}

Enumerated lists are marked with numbers or letters:

\begin{enumerate}
\item First element
\begin{enumerate}
\item First sub-item
\item Last sub-item
\end{enumerate}
\item Second element
\end{enumerate}

You can have ordered lists with jumping numbers:

\begin{enumerate}
\setcounter{enumi}{0}
\item First
\setcounter{enumi}{1}
\item Second
\setcounter{enumi}{4}
\item Jump to 5th
\end{enumerate}

\subsubsection*{Definition lists}
\label{sec:org4901476}

\begin{description}
\item[{Definition list}] List containing definitions.

\item[{Term to define}] Explication of the term.
\end{description}

\subsubsection*{Checkboxes}
\label{sec:org6814e34}
\begin{itemize}
\item[{$\square$}] First item not checked
\item[{$\boxminus$}] Second item half done
\begin{itemize}
\item[{$\square$}] Another first
\item[{$\square$}] Another second
\end{itemize}
\item[{$\boxtimes$}] Third item checked
\end{itemize}

\subsection*{Miscellaneous effects}
\label{sec:org63cc42a}

\subsubsection*{Include Org files}
\label{sec:orgd12317e}

You can include another Org file and skip its title by using the \texttt{:lines} argument
to \texttt{\#+INCLUDE}:

\begin{verbatim}
#+INCLUDE: chapter1.org :lines "2-"
\end{verbatim}

\begin{note}
File inclusion, through INCLUDE keywords, is an \textbf{export-only feature}.
\end{note}

\subsubsection*{Inline HTML}
\label{sec:orgcc336bf}

You can include raw HTML in your Org documents and it will get kept as HTML
when it's exported. XXX

Text can be preformatted (in a fixed-width font).
It is especially useful for more advanced stuff like images or tables where you
need more control of the HTML options than Org mode actually gives you.

Similarly, you can incorporate JS or do anything else you can do in a Web page
(such as importing a CSS file).

You can create named classes (to get style control from your CSS) with:

\begin{verbatim}
#+begin_info
*Info example* \\
Did you know...
#+end_info
\end{verbatim}

You can also add interactive elements to the HTML such as interactive R plots.

Finally, you can include an HTML file verbatim (during export) with:

\begin{verbatim}
#+INCLUDE: file.html html
\end{verbatim}

Don't edit the exported HTML file!

\subsubsection*{Inline \LaTeX{}}
\label{sec:org140a4cc}

You can also use raw \LaTeX{}. XXX

Text can be preformatted (in a fixed-width font).

\subsubsection*{Centered text}
\label{sec:orgd21eef9}

\begin{center}
This text is centered!
\end{center}

\subsection*{Code blocks}
\label{sec:org545dbcb}

\subsubsection*{Line numbers}
\label{sec:org901f276}

Both in \texttt{example} and in \texttt{src} snippets, you can add a \texttt{-n} switch to the end of the
\texttt{begin} line, to get the lines of the example numbered.

\begin{verbatim}
1  (defun org-xor (a b)
2    "Exclusive or."
\end{verbatim}

If you use a \texttt{+n} switch, the numbering from the previous numbered snippet will
be continued in the current one:

\begin{verbatim}
3  (if a (not b) b))
\end{verbatim}

In literal examples, Org will interpret strings like \texttt{(ref:name)} as labels, and
use them as targets for special hyperlinks like \texttt{[[(name)]]} (i.e., the reference
name enclosed in single parenthesis).  In HTML, hovering the mouse over such
a link will remote-highlight the corresponding code line, which is kind of
cool.

You can also add a \texttt{-r} switch which removes the labels from the source code.
With the \texttt{-n} switch, links to these references will be labeled by the line
numbers from the code listing, otherwise links will use the labels with no
parentheses.  Here is an example:

\begin{verbatim}
1  (save-excursion                  ;
2    (goto-char (point-min)))       ;
\end{verbatim}

In line 1, we remember the current position.  Line 2 jumps to
\texttt{point-min}.

\subsubsection*{Output}
\label{sec:orgb692c10}

The output from the \textbf{execution} of programs, scripts or commands can be inserted
in the document itself, allowing you to work in the \emph{reproducible research}
mindset.

\paragraph*{Text}
\label{sec:orgdc4ff8f}

A one-liner result:

\begin{verbatim}
date +"%Y-%m-%d"
\end{verbatim}

\begin{verbatim}
2014-03-15
\end{verbatim}

\paragraph*{Graphics}
\label{sec:org52fb2ad}

Data to be charted:

\begin{table}[htbp]
\label{tab:org846b394}
\centering
\begin{tabular}{rr}
Month & Degrees\\[0pt]
\hline
1 & 3.8\\[0pt]
2 & 4.1\\[0pt]
3 & 6.3\\[0pt]
4 & 9.0\\[0pt]
5 & 11.9\\[0pt]
6 & 15.1\\[0pt]
7 & 17.1\\[0pt]
8 & 17.4\\[0pt]
9 & 15.7\\[0pt]
10 & 11.8\\[0pt]
11 & 7.7\\[0pt]
12 & 4.8\\[0pt]
\end{tabular}
\end{table}

Code:

\begin{verbatim}
plot(data, type="b", bty="l", col=c("#ABD249"), las=1, lwd=4)
grid(nx=NULL, ny=NULL, col=c("#E8E8E8"), lwd=1)
legend("bottom", legend=c("Degrees"), col=c("#ABD249"), pch=c(19))
\end{verbatim}

The resulting chart:

\begin{center}
\includegraphics[width=.9\linewidth]{../../images/Rplot.png}
\end{center}

\paragraph*{R code block}
\label{sec:orgb294c4a}

\begin{verbatim}
library(ggplot2)
summary(cars)
\end{verbatim}

Plot:

\begin{verbatim}
library(ggplot2)
qplot(speed, dist, data = cars) + geom_smooth()
\end{verbatim}

\subsection*{Inline code}
\label{sec:org99ae7a0}

You can also evaluate code inline as follows: 1 + 1 is .

\subsection*{Notes at the footer}
\label{sec:orgb255297}

It is possible to define named footnotes\footnote{Extensively used in large documents.}, or ones with
automatic anchors\footnote{Lorem ipsum dolor sit amet, consectetur adipisicing elit, sed do
eiusmod tempor incididunt ut labore et dolore magna aliqua. Ut enim ad minim
veniam, quis nostrud exercitation ullamco laboris nisi ut aliquip ex ea
commodo consequat. Duis aute irure dolor in reprehenderit in voluptate velit
esse cillum dolore eu fugiat nulla pariatur. Excepteur sint occaecat cupidatat
non proident, sunt in culpa qui officia deserunt mollit anim id est laborum.}.

\subsection*{Formatting text}
\label{sec:org30596a7}

\subsubsection*{Text effects}
\label{sec:org5c40bc8}

\emph{Emphasize} (italics), \textbf{strongly} (bold), and \textbf{\emph{very strongly}} (bold italics).

Markup elements could be nested: this is \emph{italic text which contains
\uline{underlined text} within it}, whereas \uline{this is normal underlined text}.

Markup can span across multiple lines, by default \textbf{no more than 2}:

\textbf{This
is not
bold}

Other elements to use sparingly are:
\begin{itemize}
\item monospaced typewriter font for \texttt{inline code}
\item monospaced typewriter font for \texttt{verbatim text}
\item \sout{deleted} text (vs. \uline{inserted} text)
\item text with\textsuperscript{superscript} (for example: \texttt{m/s\textasciicircum{}\{2\}} gives m/s\textsuperscript{2})
\item text with\textsubscript{subscript} (for example: \texttt{H\_\{2\}O} gives H\textsubscript{2}O)
\end{itemize}

\subsubsection*{Quotations}
\label{sec:org8d518f1}

Use the \texttt{quote} block to typeset quoted text.

\begin{quote}
Let us change our traditional attitude to the construction of programs:
Instead of imagining that our main task is to instruct a computer what to do,
let us concentrate rather on explaining to human beings what we want a
computer to do.

The practitioner of literate programming can be regarded as an essayist, whose
main concern is with exposition and excellence of style. Such an author, with
thesaurus in hand, chooses the names of variables carefully and explains what
each variable means. He or she strives for a program that is comprehensible
because its concepts have been introduced in an order that is best for human
understanding, using a mixture of formal and informal methods that reinforce
each other.

--- Donald Knuth
\end{quote}

A short one:

\begin{quote}
Everything should be made as simple as possible,
but not any simpler -- Albert Einstein
\end{quote}

In a \texttt{verse} environment, there is an implicit line break at the end of each
line, and indentation and vertical space are preserved:

\begin{verse}
Everything should be made as simple as possible,\\[0pt]
but not any simpler -- Albert Einstein\\[0pt]
\end{verse}

Typically used for quoting passages of an email message:

\begin{verse}
>> This is an email message with "nested" quoting. Lorem ipsum dolor sit amet,\\[0pt]
>> consectetuer adipiscing elit. Aliquam hendrerit mi posuere lectus.\\[0pt]
>> Vestibulum enim wisi, viverra nec, fringilla in, laoreet vitae, risus.\\[0pt]
>\\[0pt]
> Donec sit amet nisl. Aliquam semper ipsum sit amet velit. Suspendisse id sem\\[0pt]
> consectetuer libero luctus adipiscing.\\[0pt]
\vspace*{1em}
Itemized or unordered lists (\texttt{ul}):\\[0pt]
- This is the first list item.\\[0pt]
- This is the second list item.\\[0pt]
\vspace*{1em}
Enumerated or ordered Lists (\texttt{ol}):\\[0pt]
1. This is the first list item.\\[0pt]
2. This is the second list item.\\[0pt]
\vspace*{1em}
Maybe an equation here?\\[0pt]
\vspace*{1em}
See \url{http://www.google.com/} for more information\ldots{}\\[0pt]
\end{verse}

\subsubsection*{Spaces}
\label{sec:orga4c2a0e}

Using non-breaking spaces.

Insert the Unicode character \texttt{00A0} to add a non-breaking space. FIXME
Or add/use an Org entity?

\subsection*{Mathematical formulae}
\label{sec:orge26de15}

You can embed \LaTeX{} math formatting in Org mode files using the following
syntax:

\begin{itemize}
\item For \textbf{inline math} expressions, use \texttt{\textbackslash{}(...\textbackslash{})}: \(x^2\) or \(1 < 2\).

It's \emph{not} advised to use the constructs \texttt{\$...\$} (both for Org and MathJax).

\item Centered display equation (the \emph{Euler theorem}):

\[
  \int_0^\infty e^{-x^2} dx = {{\sqrt{\pi}} \over {2}}
  \]

The use of \texttt{\textbackslash{}[...\textbackslash{}]} is for mathematical expressions which you want to make
\textbf{stand out, on their own lines}.

\LaTeX{} allows to inline such \texttt{\textbackslash{}[...\textbackslash{}]} constructs (\emph{quadratic formula}):
\[ \frac{-b \pm \sqrt{b^2 - 4 a c}}{2a} \]

\textbf{Double dollar signs (\texttt{\$\$}) should not be used}.

\item The \emph{sinus theorem} can then be written as the equation:

\begin{equation}
\label{eqn:sinalpha}
\frac{\sin\alpha}{a}=\frac{\sin\beta}{b}
\end{equation}

\item See Equation \ref{eq:orgf7947f0},

\begin{equation}
\label{eq:orgf7947f0}
n_{i+1} = \frac{n_{i} (d-i) (e-1)}{(i+1)}
\end{equation}

Only captioned equations are numbered

\item Other alternative: use \begin{equation*} or \begin{displaymath} (= the verbose
form of the \texttt{\textbackslash{}[...\textbackslash{}]} construct). M-q does not fill those.
\end{itemize}

Differently from \$\ldots{}\$ and \(...\), an equation environment produces a \textbf{numbered}
equation to which you can add a label and reference the equation by (label)
name in other parts of the text. This is not possibly with unnumbered math
environments (\$\$, \ldots{}).

\subsection*{Special characters}
\label{sec:org8ce5af5}

Some of the widely used special characters (converted from text characters to
their typographically correct entitites):

\subsubsection*{Accents}
\label{sec:orgedafc60}

\`{A} \'{A}

\subsubsection*{Punctuation}
\label{sec:org3661bf6}

Dash: -- ---

Marks: !` ?`

Quotations: \guillemotleft{} \guillemotright{}

Miscellaneous: \P{} \textordfeminine{}

\subsubsection*{Commercial symbols}
\label{sec:orga0d2175}

Property marks: \textcopyright{} \textregistered{}

Currency: \textcent{} \texteuro{} \textyen{} \pounds{}

\subsubsection*{Greek characters}
\label{sec:org04d3b73}

The Greek letters \(\alpha\), \(\beta\), and \(\gamma\) are used to denote angles.

\subsubsection*{Math characters}
\label{sec:org8bbe71c}

Science: \textpm{} \textdiv{}

Arrows: \(\to\) \(\rightarrow\) \(\leftarrow\) \(\leftrightarrow\) \(\Rightarrow\) \(\Leftarrow\) \(\Leftrightarrow\)

Function names: \(\arccos\) \(\cos\)

Signs and symbols: \textbullet{} \(\star\)

\subsubsection*{Misc}
\label{sec:org84a8986}

Suits: \(\clubsuit\) \(\spadesuit\)

\subsection*{Comments}
\label{sec:org9bbbd39}

It's possible to add comments in the document.

\subsection*{Tables}
\label{sec:org24f38e9}

You can create tables with an optional header row (by using an horizontal line
of dashes to separate it from the rest of the table).

\begin{table}[htbp]
\caption{An example of table}
\centering
\begin{tabular}{lll}
Header 1 & Header 2 & Header 3\\[0pt]
\hline
Top left & Top middle & \\[0pt]
 &  & Right\\[0pt]
Bottom left & Bottom middle & \\[0pt]
\end{tabular}
\end{table}

Columns are automatically aligned:

\begin{itemize}
\item Number-rich columns to the right, and
\item String-rich columns to the left.
\end{itemize}

If you want to override the automatic alignment, use \texttt{<r>}, \texttt{<c>} or \texttt{<l>}.

\begin{table}[htbp]
\caption{Table with alignment}
\centering
\begin{tabular}{rcl}
1 & 2 & 3\\[0pt]
right & center & left\\[0pt]
xxxxxxxxxxxx & xxxxxxxxxxxx & xxxxxxxxxxxx\\[0pt]
\end{tabular}
\end{table}

Placement:

\begin{tabular}{rr}
a & b\\[0pt]
1 & 2\\[0pt]
\end{tabular}

XXX
Different from the following:

\begin{center}
\begin{tabular}{rr}
a & b\\[0pt]
1 & 2\\[0pt]
\end{tabular}
\end{center}

\subsubsection*{Align tables on the page}
\label{sec:org4550c3d}

Here is a table on the left side:

\noindent
\begin{tabular}{rrr}
a & b & c\\[0pt]
\hline
1 & 2 & 3\\[0pt]
4 & 5 & 6\\[0pt]
\end{tabular}
\hfill

The noindent just gets rid of the indentation of the first line of a paragraph
which in this case is the table. The hfill adds infinite stretch after the
table, so it pushes the table to the left.

Here is a centered table:

\begin{center}
\begin{tabular}{rrr}
a & b & c\\[0pt]
\hline
1 & 2 & 3\\[0pt]
4 & 5 & 6\\[0pt]
\end{tabular}
\end{center}

And here's a table on the right side:

\hfill
\begin{tabular}{rrr}
a & b & c\\[0pt]
\hline
1 & 2 & 3\\[0pt]
4 & 5 & 6\\[0pt]
\end{tabular}

Here the hfill adds infinite stretch before the table, so it pushes the table
to the right.

\subsection*{Images, video and audio}
\label{sec:org57ef8ef}

\subsubsection*{Images}
\label{sec:orgeef4943}

You can insert \textbf{image} files of different \textbf{formats} to a page:

\begin{center}
\begin{tabular}{lll}
 & HTML & PDF\\[0pt]
\hline
gif & yes & \\[0pt]
jpeg & yes & \\[0pt]
png & yes & \\[0pt]
bmp & (depends on browser support) & \\[0pt]
\end{tabular}
\end{center}

In-line picture:

\begin{figure}[htbp]
\centering
\includegraphics[width=0.25\linewidth]{../../images/org-mode-unicorn.png}
\caption{Org mode logo}
\end{figure}

Direct link to just the \href{org-mode-unicorn.png}{Unicorn picture file}.

XXX Available HTML image tags include:

\begin{itemize}
\item align
\item border
\item bordercolor
\item hspace
\item vspace
\item width
\item height
\item title
\item alt
\end{itemize}

Place images side by side: XXX

\subsubsection*{Video}
\label{sec:org0026f1a}

Videos can't be added directly but you can add an image with a link to the video like this:

\href{http://img.youtube.com/vi/YOUTUBE\_VIDEO\_ID\_HERE/0.jpg}{http://www.youtube.com/watch?v=YOUTUBE\textsubscript{VIDEO}\textsubscript{ID}\textsubscript{HERE}}

\subsubsection*{Sounds}
\label{sec:orgf6ec72c}

\subsection*{Special text boxes}
\label{sec:orga064f10}

Simple box ("inline task"): XXX

\subsubsection*{Example}
\label{sec:orgc3a76df}

You can have \texttt{example} blocks.

Find entries with an \textbf{exact phrase} -- To do this, put the phrase in quotes:

\begin{verbatim}
"hd ready"
\end{verbatim}


You can create several other boxes (\texttt{info}, \texttt{tip}, \texttt{note} or \texttt{warning}) which all have
a different default image.

\subsubsection*{Info}
\label{sec:org9d7d3a2}

An info box is displayed as follows:

\begin{info}
\textbf{Info example} \\[0pt]
Did you know\ldots{}
\end{info}

\subsubsection*{Tip}
\label{sec:org52bb001}

A tip box is displayed as follows:

\begin{tip}
\textbf{Tip example} \\[0pt]
Try doing it this way\ldots{}
\end{tip}

\subsubsection*{Note}
\label{sec:org2dd3f0a}

A note box is displayed as follows:

\begin{note}
\textbf{Note example} \\[0pt]
This is a useful note\ldots{}
\end{note}

\subsubsection*{Warning}
\label{sec:org2ff8458}

A warning box is displayed as follows:

\begin{warning}
\textbf{Warning example} \\[0pt]
Be careful!  Check that you have\ldots{}
\end{warning}

\subsection*{Links}
\label{sec:org3dabed7}
\subsubsection*{Anchors}
\label{sec:orgef6bed0}
Links generally point to an headline.

They can also point to a link anchor \label{orgc288a92} in the current
document or in another document.

\subsubsection*{Hyperlinks}
\label{sec:org6b5ec84}

This document is available in \href{example.txt}{plain text}, \href{example.html}{HTML} and \href{example.pdf}{PDF}.

The links are delimited by \texttt{[square brackets]}.

\paragraph*{Internal links}
\label{sec:org56ad81f}

See:
\begin{itemize}
\item chapter \hyperref[sec:org3dabed7]{Links}
\item section \hyperref[sec:orgef6bed0]{Anchors}
\item \hyperref[orgc288a92]{target in the document}
\end{itemize}

\paragraph*{External links}
\label{sec:orgc09ee30}

See the \href{http://orgmode.org/}{Org mode Web site}.

\href{mailto:concat.fni.at-sign.pirilampo.org}{Mailto link}

\section*{Org miscellaneous}
\label{sec:org892a90a}

\subsection*{Dates}
\label{sec:org33b6293}

Timestamps: \textit{[2014-01-16 Thu] } and \textit{<2014-01-16 Thu>}.

\subsection*{{\bfseries\sffamily DONE} Buy GTD book\hfill{}\textsc{online}}
\label{sec:org3b32805}
By default, \texttt{DONE} actions will be collapsed.

Note that I should probably implement that default behavior only for \texttt{ARCHIVE}'d
items.

\subsection*{{\bfseries\sffamily TODO} Read GTD book}
\label{sec:org5494720}
\noindent\textbf{SCHEDULED:} \textit{<2014-09-11 Thu>}\\[0pt]

By default, \textbf{all} (active) entries will be expanded at page load, so that their
contents is visible.

That can be changed by adding such a line (into your Org document):

\begin{verbatim}
#+HTML_HEAD: <script> var HS_STARTUP_FOLDED = true; </script>
\end{verbatim}

\subsection*{{\bfseries\sffamily TODO} Apply GTD methodoloy}
\label{sec:org8e76fd8}
\noindent\textbf{DEADLINE:} \textit{<2014-12-01 Mon>}\\[0pt]
This section will be collapsed when loading the page because the entry has the
value \texttt{hsCollapsed} for the property \texttt{:HTML\_CONTAINER\_CLASS:}.

Powerful, no?

\subsection*{Some note\hfill{}\textsc{computer:write}}
\label{sec:orgbd033b0}

You can add tags to any entry, and hightlight all entries having some specific
tag by clicking on the buttons made accessible to you in the "Dashboard".

\subsection*{Weekly review\hfill{}\textsc{computer}}
\label{sec:org56fbb5d}

Now, you can even make your weekly review in the HTML export\ldots{} Press the \texttt{r} key
to start entering the "review mode" where all but one active entry are
collapsed, so that you can really focus on one item at a time!

\section*{Org macros}
\label{sec:orge8367f0}





Find more macros on \href{https://github.com/fniessen/org-macros}{GitHub}.

\section*{BigBlow addons}
\label{sec:org6b27c78}

The string \texttt{fixme} (in \textbf{upper case}) gets replaced by a "Fix Me!" image:

FIXME Delete this\ldots{}
\end{document}
